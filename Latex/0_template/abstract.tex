The last years there is a wide interest in \acp{UAV} which can be attributed to their low cost and wide range of use in recreational, commercial and scientific applications. Despite the large increase in drones, \ac{UAV} flights are permitted only in secluded areas. In order to be granted access to public areas, it must prove its capacity to sense and safely avoid collisions with other obstacles. Therefore, the need for a secure and reliable \ac{CAS} is imperative. In this thesis a low-cost, low computationally demanding, stereo-based, robust \ac{CAS} solution for small \acp{UAV} is designed, assuming flights primarily in an outdoor environment. In order to address this problem, firstly the existing dense stereo open-source algorithms are reviewed based on their suitability for obstacle avoidance and their computational complexity. Based on a semantic evaluation and profiling, the review concludes that \ac{BM} should be preferred for low-cost obstacle avoidance and \ac{SGBM} should be used only in highly textureless environments. Subsequently, since the imperfect accuracy of any existing stereo solution is a fact, a machine learning method is introduced in order to predict the uncertainty of the stereo measurements. This so called ``uncertainty map" method assigns an uncertainty value to every image pixel. It was shown that it can successfully predict the uncertainties of \ac{BM} and \ac{SGBM} stereo algorithms. Furthermore, a low-cost collision avoidance method was proposed which makes use of uncertainty map in sensing filtering, collision detection and path planning. The evaluation showed that the use of uncertainty map improves both collision detection and path planning, especially when \ac{BM} is used. Last but not least, the whole \ac{CAS} was implemented in an embedded system Raspberry Pi 3 model B+. Results show that real-time execution of the propsoed \ac{CAS} with a runtime frequency of 4-54 Hz is possible when \ac{BM} is used and 1-16 Hz when \ac{SGBM} is used.